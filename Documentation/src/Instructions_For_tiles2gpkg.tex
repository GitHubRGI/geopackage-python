%   Copyright (C) 2014 Reinventing Geospatial, Inc
%
%   This program is free software: you can redistribute it and/or modify
%   it under the terms of the GNU General Public License as published by
%   the Free Software Foundation, either version 3 of the License, or
%   (at your option) any later version.
%
%   This program is distributed in the hope that it will be useful,
%   but WITHOUT ANY WARRANTY; without even the implied warranty of
%   MERCHANTABILITY or FITNESS FOR A PARTICULAR PURPOSE.  See the
%   GNU General Public License for more details.
%
%   You should have received a copy of the GNU General Public License
%   along with this program.  If not, see <http://www.gnu.org/licenses/>,
%   or write to the Free Software Foundation, Inc., 59 Temple Place -
%   Suite 330, Boston, MA 02111-1307, USA.

\documentclass{article}

\usepackage{multirow}
\usepackage[margin=1in]{geometry}
\usepackage[hidelinks]{hyperref}
\usepackage{listings}
\usepackage{courier}
\usepackage{fancyhdr}

\lstset{basicstyle=\ttfamily}

\pagestyle{fancy}
\rfoot{Version Number: 1.0}

\title{Instructions For tiles2gpkg\_parallel.py}
\author{Steven D. Lander, Reinventing Geospatial}

\begin{document}
\maketitle

\section{Purpose}
This document covers installation of dependencies, in-depth explanations of
command-line arguments, and usage examples for tiles2gpkg\_parallel.py.

The file tiles2gpkg\_parallel.py is a Python script that accepts a folder of
tiles in TMS format (z/x/y) and outputs an Open Geospatial Consortium (OGC)-
compliant Geopackage.  The script leverages as much hardware capability as
possible in order package this data.

For more information about the OGC Specification for Geopackage, please visit
the Open Geospatial Consortium Website:
\url{http://www.opengeospatial.org/standards/geopackage}.

\section{Installing Dependencies}
Tiles2gpkg\_parallel.py was written for Python 2.7.x and has not been tested on
any other version.  It will run on either 32 or 64 bit systems with no issues.

The script relies on the Python Imaging Library (PIL) in order to allow
fine-grain control of the MIME type (PNG/JPEG) of each individual tile that
resides within the output Geopackage.  Without PIL, the script will simply
detect the image type of the input tile and maintain that MIME type when it
is stored to the Geopackage.
\subsection{Windows}
In order to run tiles2gpkg\_parallel.py on a Windows environment, install
the following:
\begin{itemize}
    \item
        The latest version of Python 2.7 for either x86 (32 bit) or x86\_64
        (64 bit): 
        \url{https://www.python.org/downloads/windows/}.
    \item
        The Python Imaging Library pre-compilied binary installer for your
        correct Python version (2.7) AND architecture (32/64 bit):
        \url{http://www.lfd.uci.edu/~gohlke/pythonlibs/#pillow}.

        \bf{NOTE: PIL has been deprecated in favor of Pillow, a continuation of
        the project.  They are 100\% interchangable but prefer Pillow
        whenever you can.}
\end{itemize}
\subsection{Linux}
In order to run tiles2gpkg\_parallel.py on a Linux environment, instructions
will differ slightly by distribution.  Most newer Linux distributions have
Python 2.7 as an option in their default package repository, but others such as
CentOS only have older versions.  In that case, the user will need to find out
how to get Python 2.7 from a reputable source or compile it themselves. Install
the following:
\begin{itemize}
    \item
        The latest version of Python 2.7 for your system.  For Debian-based
        distributions such as Ubuntu and RedHat type the following into a
        command line: 

        \lstinline|sudo apt-get install python2.7 python2.7-dev  python-pip base-devel| 

        This will install the Python 2.7 environment plus PIP, a Python
        module manager.  PIP is necessary for the next step.
    \item
        The Python Imaging Library python module.  On Linux, PIP will
        download the source code for PIL or Pillow and then install it on
        your system.  For Debian-based distributions type the following
        into a command line:

        \lstinline|sudo pip install Pillow|

        This will install Pillow, the successor to the deprecated version
        of the Python Imaging Library.  PIP will attempt to compile the
        Pillow source code for your system using the python2.7-dev headers
        and the compilation tools contained within the base-devel package.
\end{itemize}
\section{Usage}
\subsection{Command Line Arguments}
Tiles2gpkg\_parallel.py supports additional functionality via command line
arguments provided to the script at the time it is executed.  Following is a
outline of each one and their purpose:
\\\\
\begin{tabular}{ | c | p{14cm} | }
    \hline
    -h & Print the listing of commands available for the script.\\
    \hline
    -tileorigin & Specify the origin of the tiles contained within the input
    data folder.  Gdal2tiles.py creates tiles referenced by the bottom-left
    corner which follows TMS convention.  Other tile providers can create tiles
    with a tile origin of upper-left.  Valid options are ul, ll, nw, or sw.
    The default option is ll for lower-left.\\
    \hline
    -srs & Specify the spatial reference system of the tiles contained within
    the input data folder.  This could also be called the tile grid profile.
    Valid options are 3857 (mercator), 4326 (geodetic), and 3395 (global
    geodetic).  The default value for this field is 3857.\\
    \hline
    -imagery & Convert the MIME type of the tiles on-disk to a new type when
    they are stored in the Geopackage.  Valid options are source, mixed, png,
    and jpeg.  Specifying mixed mode will convert all tile images in the source
    folder that do not have transparency to JPEG with compression enabled for
    space savings.  Specifying source mode will preserve the file type of the
    input tile images.  The default value for this field is source.\\
    \hline
    -q & When the -imagery flag is set to either mixed or jpeg, this flag
    specifies the jpeg quality value.  Acceptable values are from 1-100
    inclusive.  Lower numbers result in smaller size images but greatly reduced
    image quality.\\
    \hline
    -T & By default, tiles2gpkg\_parallel.py takes advantage of all the
    processors available to the hardware that it is executed on.  This can mean
    that other computing tasks on the machine may suffer depending on the size
    of the packaging job. The -T flag disables this behavoir and only uses a
    single-core process to execute the job.\\
    \hline
\end{tabular}

\subsection{Examples}
\begin{itemize}
    \item
        Create a geopackage from a folder of tiles named WhiteHorse in the
        geodetic tile profile, and name the new Geopackage whitehorse.gpkg:\\
        \\
        \lstinline|python tiles2gpkg_parallel.py|\\
        \lstinline|    -srs 4326|\\
        \lstinline|    /data/tiles/geodetic/WhiteHorse /data/geopackage/whitehorse.gpkg|
    \item
        Create a geopackage from a folder of tiles named belvoir in the
        mercator profile and changing the tile images to a mix of PNG and JPEG
        images:\\
        \\
        \lstinline|python tiles2gpkg_parallel.py|\\
        \lstinline|    -srs 3857 -imagery mixed|\\
        \lstinline|    /data/tiles/mercator/belvoir /data/geopackage/belvoir-3857.gpkg|
    \item
        Create a geopackage from a folder of tiles named gnc in the world
        mercator profile with a tile origin of upper-left and also converting
        the tile images to JPEG with 50\% quality:\\
        \\
        \lstinline|python tiles2gpkg_parallel.py|\\
        \lstinline|    -srs 3395 -tileorigin ul -imagery jpeg -q 50|\\
        \lstinline|    /data/tiles/world-mercator/gnc /data/geopackage/gnc-wm.gpkg|
\end{itemize}

\section{Caveats \& Known Issues}
\begin{itemize}
    \item
        Currently it is possible to provide tiles2gpkg\_parallel.py with a
        folder of tiles with different types of data in it.  For example, an
        input folder of tiles could have tiles for the world at very high zoom
        level (0-3) but also tiles at a very low zoom level (18-19).  The
        script would only describe these tiles for the world and thus confuse a
        Geopackage viewer about where the rest of the tiles in the package are
        located.
    \item
        Currently the script does not support making a Geopackage with multiple
        raster tiles tables.
    \item
        The script only creates raster tiles at the moment and does not support
        vector features nor vector tiles.  Plans for inclusion of vector tiles
        in the future are pending.
\end{itemize}

\end{document}
